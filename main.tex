\documentclass[12pt, b5paper]{ltjsarticle}

\usepackage[T1]{fontenc}
\usepackage[utf8]{inputenc}
\usepackage[backend=biber, maxnames=100, backref=true]{biblatex}
\usepackage[binary-units = true]{siunitx}
\usepackage{amsmath, amssymb, amsthm}
\usepackage{graphicx}
\usepackage{hyperref}
\usepackage{ascmac}
\usepackage{here}

\DeclareGraphicsRule{.ai}{pdf}{.ai}{}

\title{ \LaTeX Sample1 } 

\begin{document} %document環境
    \maketitle

    \section{見出し1}
        本文はこんな感じで入力します。\par
        改行をしたいときは、字下げを行う場合は \verb+\par+ 、そうでない場合は\verb+\\+ を使います。
        \subsection{小見出し1}
            見出しの中をさらに小分けができます。こういった分け方を「章立て」と呼び、日本語では大きい単位から「章」「節」「項」と言います。\par
            \LaTeX ではデフォルトでは1.1.2、といった形で表示され、この場合は第1章第1節第1項、と呼びます。
            \subsubsection{小々見出し1}
            \LaTeX{} is a document preparation system for the \TeX{} typesetting program. It offers programmable desktop publishing features and extensive facilities for automating most aspects of typesetting and desktop publishing, including numbering and cross-referencing, tables and figures, page layout, bibliographies, and much more. \LaTeX{} was originally written in 1984 by Leslie Lamport and has become the dominant method for using \TeX; few people write in plain \TeX{} anymore. The current version is \LaTeXe.
    \section{基本的な文法}
        \subsection{箇条書き}
            箇条書きにはいくつか種類があります。
            \subsubsection{itemize}
                これは「・」が頭についた箇条書きです。
                \begin{itemize}
                    \item 箇条書き1
                    \item 箇条書き2
                    \item 箇条書き3
                \end{itemize}
            \subsubsection{enumerate}
                これは頭に数字が振られている箇条書きのコマンドです。
                \begin{enumerate}
                    \item 箇条書き1
                    \item 箇条書き2
                    \item 箇条書き3
                \end{enumerate}
            \subsubsection{description}
                これは頭につける記号をユーザが決めることができるコマンドです。
                \begin{description}
                    \item[テスト1] 箇条書き1
                    \item[テスト2] 箇条書き2
                    \item[テスト3] 箇条書き3
                \end{description}
        \subsection{表組み}
            \subsubsection{基本的な表}
                \begin{table}[H]
                    \begin{center}
                    \caption{基本的な表}
                    \begin{tabular}{|l|c|r|}
                        \hline
                        セル1 & セル2 & セル3 \\ \hline
                        セル4 & セル5 & セル6 \\ \hline
                        セル7 & セル8 & セル9 \\ \hline
                    \end{tabular} 
                    \end{center}
                \end{table}
            \subsubsection{セルの結合}
                \begin{table}[H]
                    \begin{center}
                    \caption{セルの結合}
                    \label{tab:sample1}
                    \begin{tabular}{|l|c|r|}
                        \hline
                         & \multicolumn{2}{|c|}{セル1} \\ \cline{2-3}
                        セル2 & セル3 & セル4  \\ \cline{2-2}
                         & セル5 &  \\ \hline
                    \end{tabular} 
                    \end{center}
                \end{table}
                表を参照するときはこうします(表\ref{tab:sample1})
            \subsubsection{複数の表}
            \begin{table}[H]
                \begin{center}
                \begin{tabular}{cc}
                
                \begin{minipage}{0.5\hsize}
                \begin{center}
                \begin{tabular}{|c|c|c|}
                \hline
                a & b & c \\ \hline
                d & e & f \\ \hline
                g & h & i \\ \hline
                \end{tabular}
                \caption{aが9つ}
                \end{center}
                \end{minipage}
                
                \begin{minipage}{0.5\hsize}
                \begin{center}
                \begin{tabular}{|c|c|c|}
                \hline
                j & k & l \\ \hline
                m & n & o \\ \hline
                p & q & r \\ \hline
                \end{tabular}
                \caption{bが9つ}
                \end{center}
                \end{minipage}
                
                \end{tabular}
                \end{center}
            \end{table}
        \subsection{数式}
            \subsubsection{その1}
                \begin{eqnarray}
                    G_{-1}(x,y)=x^{n-1}+x^n-1
                    \label{eq:sample1}
                \end{eqnarray}
                    
                \begin{eqnarray}
                    W(D)&=&(I(D),(1+D)I(D)) \nonumber \\
                    &=&I(D)[1,1+D] \nonumber \\
                    &=&I(D)G(D)
                \end{eqnarray} 
                
                \[
                x=a+b+c
                \]
                
                本文中にも数式を$ x=a+b+c $表示できます。\par
                数式の参照はこうやります(式(\ref{eq:sample1}))            
            \subsubsection{その2}
                \begin{eqnarray}
                    y=\frac{\sqrt[2]{1+x}}{1-x}
                    \end{eqnarray} 
                    
                    \begin{eqnarray}
                    \left(\frac{1+x}{1-x}\right) +(\frac{1-x}{1+x})
                    \end{eqnarray} 
                    
                    \begin{eqnarray}
                    c_{k,l}=\left\{ \begin{array}{ll}
                    1[\mathrm{m/s]}] & (l=k) \\
                    \alpha & (|l-k|=1) \\
                    0[\mathrm{m/s]} & (上記以外) \\
                    \end{array} \right.
                \end{eqnarray}
        \subsection{画像の挿入}
            \begin{figure}[H]
                \begin{center}
                %\includegraphics[width=5cm]{figure1.jpg}
                \end{center}
                \caption{図の説明}
                \label{fig:sample1}
            \end{figure}
            画像の参照はこうやります(図\ref{fig:sample1})
    \section{参考文献の参照}
        参考文献を参照する場合は\cite{bib1}こうします。
    \section{その他のコマンド}
        その他特殊文字や細かい操作方法などは\cite{sheat1,sheat2}を参照してください。

    \begin{thebibliography}{9}
        \bibitem{bib1} ここに参考文献を入力します
        \bibitem{sheat1} unknown, `` LaTeXコマンドシート一覧,'' 2003, [online] Available: \url{http://www002.upp.so-net.ne.jp/latex/index.html}
        \bibitem{sheat2} unknown, `` LaTeXコマンド集,'' 2009, [online] Available: \url{http://www.latex-cmd.com}
    \end{thebibliography}
\end{document}

